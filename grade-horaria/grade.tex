\documentclass[12pt, border=10pt]{standalone}

% --- PACOTES NECESSÁRIOS ---
\usepackage[T1]{fontenc}
\usepackage[brazil]{babel}
\usepackage{lmodern}
\usepackage{tikz}
\usepackage{tikz-timetable} % O pacote principal para a grade de horários

% --- DEFINIÇÃO DE CORES (Opcional) ---
\definecolor{aulaSegunda}{RGB}{227, 235, 255} % Um azul claro
\definecolor{aulaQuarta}{RGB}{255, 230, 230} % Um vermelho/rosa claro

\begin{document}

% --- 1. DEFINIÇÃO DA ESTRUTURA DA GRADE ---
% Criamos um novo tipo de grade chamada 'meuHorario'
\newtimetable{meuHorario}

% Definimos as horas que aparecerão na coluna da esquerda.
% 'start' é a primeira hora, 'end' é a última, 'step' é o intervalo.
\sethours{start=8, end=23, step=1}

% Definimos os dias da semana que serão as colunas.
% O primeiro argumento é a largura da coluna.
\setdays{4.5cm}{SEGUNDA, QUARTA}

% --- 2. DESENHANDO A GRADE E ADICIONANDO AS AULAS ---
\begin{timetable}[
    % Opções de Estilo:
    linecolor=gray!40,      % Cor das linhas da grade
    headercolor=gray!20,    % Cor do cabeçalho
    bodyfs=\bfseries\sffamily, % Fonte para o corpo (negrito, sans-serif)
    headerfs=\Large\bfseries\sffamily % Fonte para o cabeçalho
    ]

    % --- Adicionando as aulas ---
    % \event{dia}{hora_inicio}{hora_fim}{Nome da Aula}[cor]
    
    % Aula de Segunda-Feira
    \event{1}{13:30}{17:10}{
        \centering % Centraliza o texto dentro do bloco
        7600024 - \\ 
        Laboratório Avançado\\ 
        de Física I
    }[bgcolor=aulaSegunda]

    % Aulas de Quarta-Feira
    \event{2}{19:00}{20:40}{
        \centering 
        SME0828 - \\ 
        Introdução à\\ 
        Ciência de Dados
    }[bgcolor=aulaQuarta]
    
    % Segunda aula da Segunda-Feira
     \event{1}{21:00}{22:40}{
        \centering 
        SME0828 - \\ 
        Introdução à\\ 
        Ciência de Dados
    }[bgcolor=aulaSegunda]

\end{timetable}

\end{document}