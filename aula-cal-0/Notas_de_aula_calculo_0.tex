\documentclass[12pt, a4paper]{article} %determina o tamanho da fonte, o tipo de papel e o tipo de documento.

\setlength{\parindent}{1.0 cm} %tamanho do espaço para começar o parágrafo.
\setlength{\parskip}{0.5cm} %tamanho do espaço entre os parágrafos.

%Aqui ficam os pacotes utilizados para formatação do documento de modo geral:

\usepackage[utf8]{inputenc} 
\usepackage{indentfirst} %Coloca espaços nos inícios de parágrafos automaticamente. 
\usepackage[brazilian]{babel} %
\usepackage{amsmath}
\usepackage[hmargin=3cm, vmargin=2.5cm, bmargin=2.5cm]{geometry}
\usepackage{multicol}
\usepackage{graphicx} %para poder inserir imagens
\usepackage{subfig}
\usepackage{booktabs} 
\usepackage{hyperref} %para poder adicionar links e hiperlinks
\usepackage{float} %para poder posicionar as imagens


\usepackage{listings} %para poder incluir códigos
\usepackage{xcolor}
\definecolor{codegreen}{rgb}{0,0.6,0}
\definecolor{codegray}{rgb}{0.5,0.5,0.5}
\definecolor{codepurple}{rgb}{0.58,0,0.82}
\definecolor{backcolour}{rgb}{0.95,0.95,0.92}

\begin{document} %começa alguma coisa,neste caso, o documento, sempre importante lembrar de colocar o \end{} para não dar erro 
	
	\begin{titlepage}
		\begin{center}
\Huge{Universidade de São Paulo}\\
\large{Instituto de Física de São Carlos}\\
\vspace{20pt}
\vspace{200pt}
\textbf{Notas de Aula C\'alculo 0: Revis\~ao P1 de F\'isica I}\\
\vspace{2cm}
Pedro Calligaris Delbem\\
\end{center}

		\begin{center}
			\vspace{\fill}
	Abril de 2025	
		\end{center}
	\end{titlepage}

%####################################################################### SUMÁRIO
	\tableofcontents 
	\thispagestyle{empty}
	\newpage
%#########################################################################

\section{Movimento em Duas e Três Dimensões}

    \subsection{Aplica\c{c}\~oes de Vetores na F\'isica}

        Vetores s\~ao grandezas f\'isicas que possuem m\'agnitude e dire\c{c}\~ao.
        Em f\'isica, os vetores s\~ao utilizados para representar grandezas como:
        \begin{itemize}
            \item For\c{c}a
            \item Velocidade
            \item Acelera\c{c}\~ao
            \item Deslocamento
        \end{itemize}
        entre outras.

        Vejamos um exerc\'icio pr\'atico para entender melhor o conceito de vetores.
        \subsection{Exerc\'icio 1}
            Um rio tem largura $L=0.76Km$ e uma correnteza, paralela \'a margem, com velocidade de $v_{corr}=4km/h$. Um barco tem rapidez m\'axima de $v_{barco}=4m/s$ em \'aguas paradas. Determine o \^angulo de inclina\c{c}\~ao do barco em rela\c{c}\~ao ao rio para que o mesmo atravesse o rio em linha reta.

            \textbf{Desenhando:}

            \textbf{Solu\c{c}\~ao:}


      
    \subsection{Exerc\'icio 2}

     
\section{Leis de Newton}

    \subsection{Exerc\'icio 1}


    \subsection{Exerc\'icio 2}


    \subsection{Exerc\'icio 3}

       
\end{document}